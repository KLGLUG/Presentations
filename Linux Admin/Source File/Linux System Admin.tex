\documentclass{beamer}

\usepackage[british]{babel}
\usepackage{graphicx,hyperref,ru,url}

% The title of the presentation:
%  - first a short version which is visible at the bottom of each slide;
%  - second the full title shown on the title slide;
\title[GNU/Linux System Administration]{
  GNU/Linux System Administration }

% Optional: a subtitle to be dispalyed on the title slide
\subtitle{}

% The author(s) of the presentation:
%  - again first a short version to be displayed at the bottom;
%  - next the full list of authors, which may include contact information;
\author[Copyleft]{
  \\\medskip
  Sripath Roy Koganti %\and Vamsi Krishna 
  \\\medskip
  \small {Executive Committee, Swecha \\}
  \small {Asst Prof, Dept of ECE, KL University \\}
  {\small \url{sripathroy@swecha.net}} \\ 
  {\small {+91 9963436838}}}
  
  
\institute[]{
  2 Day Workshop at : \\ 
  Department of Computer Science and Engineering \\
  JNTUK - University College of Engineering Vizianagaram}


% Add a date and possibly the name of the event to the slides
%  - again first a short version to be shown at the bottom of each slide
%  - second the full date and event name for the title slide
\date[22, 23 September 2017]{
  22, 23 September 2017}

\begin{document}

\begin{frame}[plain]
  \titlepage
\end{frame}

\begin{frame}
  \frametitle{Outline}

  \tableofcontents
\end{frame}

% Section titles are shown in at the top of the slides with the current section 
% highlighted. Note that the number of sections determines the size of the top 
% bar, and hence the university name and logo. If you do not add any sections 
% they will not be visible.
\section{Files and Folders}

\begin{frame}
  \frametitle{Files and Folders}

  \begin{tabular}{|c|c|}
\hline
    \textbf{Command} & \textbf{Usage}\\
\hline
    touch & Creates an empty file\\
\hline
	cat & Shows content of a file\\
\hline
	wc & Counts number of Lines, Words, Characters in a file\\
\hline
	mkdir & Creates a directory\\
\hline
	ls & List files\\
\hline
	sort & Sort data in a file\\
\hline
\end{tabular}

\end{frame}

\section{File Manipulation}
\begin{frame}
  \frametitle{File Manipulation}

  \begin{tabular}{|c|c|}
\hline
    \textbf{Command} & \textbf{Usage}\\
\hline
    cp & Copies one file to another file\\
\hline
	mv & Renames a file \\
\hline
	mv & Moves file from one directory to another directory\\
\hline
	rm & Deletes a file\\
\hline
	cd & Switch to a directory\\
\hline
	pwd & Shows the current directory\\
	
\hline
\end{tabular}

\end{frame}

\section{File Permissions}
\begin{frame}
  \frametitle{File Permissions}

  \begin{tabular}{|c|c|}
\hline
    \textbf{Command} & \textbf{Usage}\\
\hline
    chmod & Changes file permissions\\
\hline
	chown & Change file ownership \\
\hline
	chgrp & Change group permissions\\
\hline
	su & Super user mode\\
\hline
	sudo & Execute command as root\\

\hline
\end{tabular}

\end{frame}

\section{Text Editors and Help}
\begin{frame}
  \frametitle{Text Editors and Help}
 \begin{block}{Editors}
  \begin{itemize}
      \item gedit - GUI Editor
      \item nano - Editor with options
    \end{itemize}
	\end{block}
	\begin{block}{Help}
	\begin{itemize}
      \item man - Manual
      \item help - Manual
    \end{itemize}
\end{block}
\end{frame}


\section{Disk Management}
\begin{frame}
  \frametitle{Disk Management}

  \begin{tabular}{|c|c|}
  \hline
    \textbf{Command} & \textbf{Usage}\\
\hline
    df & Partion details \\
\hline
    mount & Mount a file system\\
\hline
	 umount & Unmount a file system\\
\hline
	du &  Disk usage in directory tree\\
\hline
	free & To see memory usage\\
\hline
\end{tabular}

\end{frame}


\section{Tasks}
\begin{frame}
  \frametitle{Tasks}
  \begin{block}{ Task 1: Permissions}
  
  \begin{itemize}
    \item Copy files a and A to Swecha folder
    \item See the permissions for file
    \item Change file permissions as all can read, only group can write and only user can execute
    \item Now check permissions of Swecha folder
    \item Change its permissions
  \end{itemize}
  \end{block}
\end{frame}

\begin{frame}
  \frametitle{Tasks}
 \begin{block}{ Task 2: Files and Folders}
  
  \begin{itemize}
    \item Create a folder Swecha
    \item Enter Swecha folder
    \item Create 2 files a and A
    \item Edit files with content
    \item Move files to Desktop
   \end{itemize}
   \end{block}
\end{frame}


\begin{frame}
  \frametitle{Tasks}
 \begin{block}{ Task 3: Shell Scripting}
  
  \begin{itemize}
    \item Open a editor
    \item Write all commands used for files and folders task
    \item Save file with .sh extension
    \item run the file
  %  \item Move files to Desktop
  \end{itemize}
  \end{block}
\end{frame}

\begin{frame}
  \frametitle{Tasks}
 \begin{block}{ Task 4: Create your Own Command}
  
  \begin{itemize}
    \item Convert the Task 3 File to Executable(Change Permissions)
    \item Rename by removing .sh extension
    \item Copy the file to /bin
    \item Type the command anywhere
  %  \item Move files to Desktop
  \end{itemize}
  \end{block}
\end{frame}


\end{document}
